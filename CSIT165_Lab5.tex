% Options for packages loaded elsewhere
\PassOptionsToPackage{unicode}{hyperref}
\PassOptionsToPackage{hyphens}{url}
%
\documentclass[
]{article}
\usepackage{amsmath,amssymb}
\usepackage{iftex}
\ifPDFTeX
  \usepackage[T1]{fontenc}
  \usepackage[utf8]{inputenc}
  \usepackage{textcomp} % provide euro and other symbols
\else % if luatex or xetex
  \usepackage{unicode-math} % this also loads fontspec
  \defaultfontfeatures{Scale=MatchLowercase}
  \defaultfontfeatures[\rmfamily]{Ligatures=TeX,Scale=1}
\fi
\usepackage{lmodern}
\ifPDFTeX\else
  % xetex/luatex font selection
\fi
% Use upquote if available, for straight quotes in verbatim environments
\IfFileExists{upquote.sty}{\usepackage{upquote}}{}
\IfFileExists{microtype.sty}{% use microtype if available
  \usepackage[]{microtype}
  \UseMicrotypeSet[protrusion]{basicmath} % disable protrusion for tt fonts
}{}
\makeatletter
\@ifundefined{KOMAClassName}{% if non-KOMA class
  \IfFileExists{parskip.sty}{%
    \usepackage{parskip}
  }{% else
    \setlength{\parindent}{0pt}
    \setlength{\parskip}{6pt plus 2pt minus 1pt}}
}{% if KOMA class
  \KOMAoptions{parskip=half}}
\makeatother
\usepackage{xcolor}
\usepackage[margin=1in]{geometry}
\usepackage{color}
\usepackage{fancyvrb}
\newcommand{\VerbBar}{|}
\newcommand{\VERB}{\Verb[commandchars=\\\{\}]}
\DefineVerbatimEnvironment{Highlighting}{Verbatim}{commandchars=\\\{\}}
% Add ',fontsize=\small' for more characters per line
\usepackage{framed}
\definecolor{shadecolor}{RGB}{248,248,248}
\newenvironment{Shaded}{\begin{snugshade}}{\end{snugshade}}
\newcommand{\AlertTok}[1]{\textcolor[rgb]{0.94,0.16,0.16}{#1}}
\newcommand{\AnnotationTok}[1]{\textcolor[rgb]{0.56,0.35,0.01}{\textbf{\textit{#1}}}}
\newcommand{\AttributeTok}[1]{\textcolor[rgb]{0.13,0.29,0.53}{#1}}
\newcommand{\BaseNTok}[1]{\textcolor[rgb]{0.00,0.00,0.81}{#1}}
\newcommand{\BuiltInTok}[1]{#1}
\newcommand{\CharTok}[1]{\textcolor[rgb]{0.31,0.60,0.02}{#1}}
\newcommand{\CommentTok}[1]{\textcolor[rgb]{0.56,0.35,0.01}{\textit{#1}}}
\newcommand{\CommentVarTok}[1]{\textcolor[rgb]{0.56,0.35,0.01}{\textbf{\textit{#1}}}}
\newcommand{\ConstantTok}[1]{\textcolor[rgb]{0.56,0.35,0.01}{#1}}
\newcommand{\ControlFlowTok}[1]{\textcolor[rgb]{0.13,0.29,0.53}{\textbf{#1}}}
\newcommand{\DataTypeTok}[1]{\textcolor[rgb]{0.13,0.29,0.53}{#1}}
\newcommand{\DecValTok}[1]{\textcolor[rgb]{0.00,0.00,0.81}{#1}}
\newcommand{\DocumentationTok}[1]{\textcolor[rgb]{0.56,0.35,0.01}{\textbf{\textit{#1}}}}
\newcommand{\ErrorTok}[1]{\textcolor[rgb]{0.64,0.00,0.00}{\textbf{#1}}}
\newcommand{\ExtensionTok}[1]{#1}
\newcommand{\FloatTok}[1]{\textcolor[rgb]{0.00,0.00,0.81}{#1}}
\newcommand{\FunctionTok}[1]{\textcolor[rgb]{0.13,0.29,0.53}{\textbf{#1}}}
\newcommand{\ImportTok}[1]{#1}
\newcommand{\InformationTok}[1]{\textcolor[rgb]{0.56,0.35,0.01}{\textbf{\textit{#1}}}}
\newcommand{\KeywordTok}[1]{\textcolor[rgb]{0.13,0.29,0.53}{\textbf{#1}}}
\newcommand{\NormalTok}[1]{#1}
\newcommand{\OperatorTok}[1]{\textcolor[rgb]{0.81,0.36,0.00}{\textbf{#1}}}
\newcommand{\OtherTok}[1]{\textcolor[rgb]{0.56,0.35,0.01}{#1}}
\newcommand{\PreprocessorTok}[1]{\textcolor[rgb]{0.56,0.35,0.01}{\textit{#1}}}
\newcommand{\RegionMarkerTok}[1]{#1}
\newcommand{\SpecialCharTok}[1]{\textcolor[rgb]{0.81,0.36,0.00}{\textbf{#1}}}
\newcommand{\SpecialStringTok}[1]{\textcolor[rgb]{0.31,0.60,0.02}{#1}}
\newcommand{\StringTok}[1]{\textcolor[rgb]{0.31,0.60,0.02}{#1}}
\newcommand{\VariableTok}[1]{\textcolor[rgb]{0.00,0.00,0.00}{#1}}
\newcommand{\VerbatimStringTok}[1]{\textcolor[rgb]{0.31,0.60,0.02}{#1}}
\newcommand{\WarningTok}[1]{\textcolor[rgb]{0.56,0.35,0.01}{\textbf{\textit{#1}}}}
\usepackage{graphicx}
\makeatletter
\def\maxwidth{\ifdim\Gin@nat@width>\linewidth\linewidth\else\Gin@nat@width\fi}
\def\maxheight{\ifdim\Gin@nat@height>\textheight\textheight\else\Gin@nat@height\fi}
\makeatother
% Scale images if necessary, so that they will not overflow the page
% margins by default, and it is still possible to overwrite the defaults
% using explicit options in \includegraphics[width, height, ...]{}
\setkeys{Gin}{width=\maxwidth,height=\maxheight,keepaspectratio}
% Set default figure placement to htbp
\makeatletter
\def\fps@figure{htbp}
\makeatother
\setlength{\emergencystretch}{3em} % prevent overfull lines
\providecommand{\tightlist}{%
  \setlength{\itemsep}{0pt}\setlength{\parskip}{0pt}}
\setcounter{secnumdepth}{-\maxdimen} % remove section numbering
\usepackage{booktabs}
\usepackage{longtable}
\usepackage{array}
\usepackage{multirow}
\usepackage{wrapfig}
\usepackage{float}
\usepackage{colortbl}
\usepackage{pdflscape}
\usepackage{threeparttable}
\usepackage{threeparttablex}
\usepackage[normalem]{ulem}
\usepackage{makecell}
\usepackage{xcolor}
\usepackage{ggrepel}
\ifLuaTeX
  \usepackage{selnolig}  % disable illegal ligatures
\fi
\usepackage{bookmark}
\IfFileExists{xurl.sty}{\usepackage{xurl}}{} % add URL line breaks if available
\urlstyle{same}
\hypersetup{
  pdftitle={Lab 5: Visualizing Coronavirus Data},
  pdfauthor={Name: Sal Figueroa},
  hidelinks,
  pdfcreator={LaTeX via pandoc}}

\title{Lab 5: Visualizing Coronavirus Data}
\author{Name: Sal Figueroa}
\date{2025-05-04}

\begin{document}
\maketitle

{
\setcounter{tocdepth}{3}
\tableofcontents
}
\subsection{Github Repository}\label{github-repository}

\emph{Holds all related files}

\href{https://github.com/Figgs0bit/CSIT165-Lab5}{github: Figgs0bit-Lab4
(https://github.com/Figgs0bit/CSIT165-Lab5.git)}

\subsection{Required data sets}\label{required-data-sets}

\emph{This lab represents data downloaded on 04/xx/2025}

\href{https://github.com/CSSEGISandData/COVID-19/tree/master/csse_covid_19_data/csse_covid_19_time_series}{Human
Proteins Data Set ()}

\subsubsection{\texorpdfstring{\emph{1. 2019 Novel Coronavirus COVID-19
(2019-nCoV) Global
Confirmations.}}{1. 2019 Novel Coronavirus COVID-19 (2019-nCoV) Global Confirmations.}}\label{novel-coronavirus-covid-19-2019-ncov-global-confirmations.}

\emph{This data set is operated by the John Hopkins University Center
for Systems Science and Engineering (JHU CSSE). Data set includes a
daily time series CSV summary table confirmed cases of COVID-19. Lat and
Long refer to coordinate references for the data field. Date fields are
stored in MM/DD/YYYY format.}

\href{https://github.com/CSSEGISandData/COVID-19/tree/master/csse_covid_19_data/csse_covid_19_time_series}{2019
Novel Coronavirus COVID-19 (2019-nCoV) Global Confirmations}

\subsubsection{\texorpdfstring{\emph{2. Human Proteins Data
Set}}{2. Human Proteins Data Set}}\label{human-proteins-data-set}

\emph{This data set is a tibble created from parsing Homo sapiens
protein fasta files curated by the Genome Reference Consortium as part
of the Human Genome Project. The original protein fasta file can be
found in NCBI, here. Data consists of two columns, Gene and
Protein.Sequence. Gene represents every gene product, or protein, made
by humans. Protein.Sequence represents the primary amino acid structure
of its correspondent gene. Each amino acid is represented as a single
capital letter and the sequence of letters is unique to each gene.}

\subsection{Instructions}\label{instructions}

\emph{Before beginning your objectives in your final document, please
state which day you downloaded the data sets on for analysis. The
objectives for this lab will cumulatively cover many subjects discussed
in this course and will also contain an objective for manipulating
strings.}

\emph{The surgeon general for the United States recently created a new
data science initiative, CSIT-165, that uses data science to
characterize pandemic diseases. CSIT-165 disseminates data driven
analyses to state governors. You are a data scientist for CSIT-165 and
it is up to you and you alone to manipulate and visualize COVID-19 data
for disease control.}

\subsection{Objective 1}\label{objective-1}

\emph{Create a scatter plot for counts per day of the the top five
confirmed countries. For this objective, please use dplyr and tidyr to
manipulate data and ggplot2 to create the visualization. Scatter plot
must have specified colors, a non-standard theme for display, and custom
a customized titles, axis labels, and legend labels.}

\begin{Shaded}
\begin{Highlighting}[]
\CommentTok{\#Ensure FinalColName is treated as a string}
\NormalTok{FinalColName }\OtherTok{\textless{}{-}} \FunctionTok{as.character}\NormalTok{(FinalColName)}

\CommentTok{\#Load Confirmed\_Global into temp variable}
\NormalTok{DF }\OtherTok{\textless{}{-}}\NormalTok{ Confirmed\_Global}

\CommentTok{\#Reorganize the data frame with the column specified in FinalColName in descending order}
\NormalTok{Decrease\_DF }\OtherTok{\textless{}{-}}\NormalTok{ DF[}\FunctionTok{order}\NormalTok{(DF[[FinalColName]], }\AttributeTok{decreasing =} \ConstantTok{TRUE}\NormalTok{),]}


\CommentTok{\#Creates data frame with the top 5 Total unaccounted cases.}
\CommentTok{\#Here the "head" command trims away any row past the 5th row. }
\NormalTok{top5\_DF }\OtherTok{\textless{}{-}} \FunctionTok{head}\NormalTok{(Decrease\_DF, }\DecValTok{5}\NormalTok{)}

\CommentTok{\#Creates a vector full of Dates reflective of the data frame dates. }
\NormalTok{Date\_Days }\OtherTok{\textless{}{-}} \FunctionTok{as.Date}\NormalTok{(}\FunctionTok{seq}\NormalTok{(}\FunctionTok{as.Date}\NormalTok{(}\StringTok{"2020/1/22"}\NormalTok{), }\FunctionTok{as.Date}\NormalTok{(}\StringTok{"2023/3/9"}\NormalTok{), }\StringTok{"days"}\NormalTok{))}


\CommentTok{\#Creates Vectors for the Top 5 Countries Data}
\CommentTok{\#Rpw 1: Country}
\NormalTok{rowOne }\OtherTok{\textless{}{-}} \FunctionTok{as.vector}\NormalTok{(}\FunctionTok{t}\NormalTok{(top5\_DF[}\DecValTok{1}\NormalTok{,]))}
\NormalTok{rowOne }\OtherTok{\textless{}{-}} \FunctionTok{as.numeric}\NormalTok{(rowOne[}\DecValTok{5}\SpecialCharTok{:}\NormalTok{MAXNameCol])}
\CommentTok{\#Rpw 2: Country}
\NormalTok{rowTwo }\OtherTok{\textless{}{-}} \FunctionTok{as.vector}\NormalTok{(}\FunctionTok{t}\NormalTok{(top5\_DF[}\DecValTok{2}\NormalTok{,]))}
\NormalTok{rowTwo }\OtherTok{\textless{}{-}} \FunctionTok{as.numeric}\NormalTok{(rowTwo[}\DecValTok{5}\SpecialCharTok{:}\NormalTok{MAXNameCol])}
\CommentTok{\#Rpw 3: Country}
\NormalTok{rowThree }\OtherTok{\textless{}{-}} \FunctionTok{as.vector}\NormalTok{(}\FunctionTok{t}\NormalTok{(top5\_DF[}\DecValTok{3}\NormalTok{,]))}
\NormalTok{rowThree }\OtherTok{\textless{}{-}} \FunctionTok{as.numeric}\NormalTok{(rowThree[}\DecValTok{5}\SpecialCharTok{:}\NormalTok{MAXNameCol])}
\CommentTok{\#Rpw 4: Country}
\NormalTok{rowFour }\OtherTok{\textless{}{-}} \FunctionTok{as.vector}\NormalTok{(}\FunctionTok{t}\NormalTok{(top5\_DF[}\DecValTok{4}\NormalTok{,]))}
\NormalTok{rowFour }\OtherTok{\textless{}{-}} \FunctionTok{as.numeric}\NormalTok{(rowFour[}\DecValTok{5}\SpecialCharTok{:}\NormalTok{MAXNameCol])}
\CommentTok{\#Rpw 5: Country}
\NormalTok{rowFive }\OtherTok{\textless{}{-}} \FunctionTok{as.vector}\NormalTok{(}\FunctionTok{t}\NormalTok{(top5\_DF[}\DecValTok{5}\NormalTok{,]))}
\NormalTok{rowFive }\OtherTok{\textless{}{-}} \FunctionTok{as.numeric}\NormalTok{(rowFive[}\DecValTok{5}\SpecialCharTok{:}\NormalTok{MAXNameCol])}

\CommentTok{\#Creates Data.frame with the dates and the top five countries.}
\CommentTok{\#The data has now been formatted as in an Inverse matrix format}
\NormalTok{T\_top5\_DF }\OtherTok{\textless{}{-}} \FunctionTok{data.frame}\NormalTok{(Date\_Days, rowOne, rowTwo, rowThree, rowFour, rowFive)}

\CommentTok{\#Renames the column names}
\FunctionTok{colnames}\NormalTok{(T\_top5\_DF) }\OtherTok{\textless{}{-}} \FunctionTok{c}\NormalTok{(}\StringTok{"Date"}\NormalTok{, top5\_DF[}\DecValTok{1}\NormalTok{,}\DecValTok{2}\NormalTok{], top5\_DF[}\DecValTok{2}\NormalTok{,}\DecValTok{2}\NormalTok{], top5\_DF[}\DecValTok{3}\NormalTok{,}\DecValTok{2}\NormalTok{], top5\_DF[}\DecValTok{4}\NormalTok{,}\DecValTok{2}\NormalTok{], top5\_DF[}\DecValTok{5}\NormalTok{,}\DecValTok{2}\NormalTok{])}

\CommentTok{\#Converts data.frame to long format: assumes T\_top5\_DF has Date + 5 country columns}
\CommentTok{\#{-}Date excludes the date here. Serves to plot multiple country data into one plot window. }
\NormalTok{T\_top5\_long }\OtherTok{\textless{}{-}}\NormalTok{ T\_top5\_DF }\SpecialCharTok{\%\textgreater{}\%}
  \FunctionTok{pivot\_longer}\NormalTok{(}\SpecialCharTok{{-}}\NormalTok{Date, }\AttributeTok{names\_to =} \StringTok{"Country"}\NormalTok{, }\AttributeTok{values\_to =} \StringTok{"Cases"}\NormalTok{)}



\CommentTok{\# Plot all 5 countries in one panel}
\FunctionTok{ggplot}\NormalTok{(T\_top5\_long, }\FunctionTok{aes}\NormalTok{(}\AttributeTok{x =}\NormalTok{ Date, }\AttributeTok{y =}\NormalTok{ Cases, }\AttributeTok{color =}\NormalTok{ Country)) }\SpecialCharTok{+} \CommentTok{\#Loads each of the top 5 couties data via color = country}

  \FunctionTok{geom\_line}\NormalTok{(}\AttributeTok{size =} \FloatTok{0.125}\NormalTok{) }\SpecialCharTok{+} \CommentTok{\#Sets thickness of line}

  \CommentTok{\#Plot labels, title, x \& y axis}
  \FunctionTok{labs}\NormalTok{(}\AttributeTok{title =} \StringTok{"COVID Data for Top 5 Countries"}\NormalTok{, }\AttributeTok{x =} \StringTok{"Date: (Year/Day/Month)"}\NormalTok{, }\AttributeTok{y =} \StringTok{"Confrimed Global Cases"}\NormalTok{) }\SpecialCharTok{+} 

  \FunctionTok{theme\_minimal}\NormalTok{(}\AttributeTok{base\_size =} \DecValTok{8}\NormalTok{) }\SpecialCharTok{+} \CommentTok{\#Creates simple plot theme w/ a size 8 font size}

  \FunctionTok{theme}\NormalTok{(}\AttributeTok{axis.text.x =} \FunctionTok{element\_text}\NormalTok{(}\AttributeTok{angle =} \DecValTok{90}\NormalTok{, }\AttributeTok{hjust =} \DecValTok{1}\NormalTok{), }\CommentTok{\#rotates x{-}axis text and hjust aligns the text to the right}
        \AttributeTok{axis.text.y =} \FunctionTok{element\_text}\NormalTok{(}\AttributeTok{angle =} \DecValTok{315}\NormalTok{, }\AttributeTok{hjust =} \DecValTok{1}\NormalTok{), }\CommentTok{\#rotates y{-}axis text and hjust aligns the text to the right}
        \AttributeTok{axis.title.y =} \FunctionTok{element\_text}\NormalTok{(}\AttributeTok{margin =} \FunctionTok{margin}\NormalTok{(}\AttributeTok{r =} \DecValTok{12}\NormalTok{)), }\CommentTok{\#adds margin to y axis}
        \AttributeTok{panel.grid.major =} \FunctionTok{element\_line}\NormalTok{(}\AttributeTok{size =} \FloatTok{0.2}\NormalTok{), }\CommentTok{\#major grid lines}
        \AttributeTok{panel.grid.minor =} \FunctionTok{element\_line}\NormalTok{(}\AttributeTok{size =} \FloatTok{0.0625}\NormalTok{) }\CommentTok{\#minor grid lines}
\NormalTok{       )}
\end{Highlighting}
\end{Shaded}

\begin{verbatim}
## Warning: Using `size` aesthetic for lines was deprecated in ggplot2 3.4.0.
## i Please use `linewidth` instead.
## This warning is displayed once every 8 hours.
## Call `lifecycle::last_lifecycle_warnings()` to see where this warning was
## generated.
\end{verbatim}

\begin{verbatim}
## Warning: The `size` argument of `element_line()` is deprecated as of ggplot2 3.4.0.
## i Please use the `linewidth` argument instead.
## This warning is displayed once every 8 hours.
## Call `lifecycle::last_lifecycle_warnings()` to see where this warning was
## generated.
\end{verbatim}

\includegraphics{CSIT165_Lab5_files/figure-latex/ob1-1.pdf}

\subsection{Objective 2}\label{objective-2}

\emph{Understanding how COVID-19 enters human cells requires that we
have a better understanding of human proteins. It has been shown that
COVID-19 is able to enter cells by binding to Angiotensin-Converting
Enzyme 2 receptors in the heart, lungs, and intestines.
Angiotensin-Converting Enzyme 2 is used by the body to regulate blood
pressure and inflammation.}

\emph{1. Show how many different isoforms of Angiotensin-Converting
Enzyme 2 humans make using str\_detect and regular expressions.}
\emph{Use the pattern ``Angiotensin-Converting Enzyme 2 isoform'' with
regular expressions to include all variations irrespective of first
letter capitalization.}

\begin{Shaded}
\begin{Highlighting}[]
\CommentTok{\#Renames human\_protiens data frame}
\NormalTok{proteins }\OtherTok{\textless{}{-}}\NormalTok{ human\_proteins}
\CommentTok{\# Check column names}
\CommentTok{\#str(proteins)}

\NormalTok{ace2\_isoforms }\OtherTok{\textless{}{-}}\NormalTok{ proteins }\SpecialCharTok{\%\textgreater{}\%}
  \FunctionTok{filter}\NormalTok{(}\FunctionTok{str\_detect}\NormalTok{(Gene, }\FunctionTok{regex}\NormalTok{(}\StringTok{"Angiotensin{-}Converting Enzyme 2 isoform"}\NormalTok{, }\AttributeTok{ignore\_case =} \ConstantTok{TRUE}\NormalTok{)))}

\FunctionTok{head}\NormalTok{(ace2\_isoforms)}
\end{Highlighting}
\end{Shaded}

\begin{verbatim}
## # A tibble: 3 x 2
##   Gene                                       Protein.Sequence                   
##   <chr>                                      <chr>                              
## 1 angiotensin-converting enzyme 2 isoform X1 MSSSSWLLLSLVAVTAAQSTIEEQAKTFLDKFNH~
## 2 angiotensin-converting enzyme 2 isoform X2 MSSSSWLLLSLVAVTAAQSTIEEQAKTFLDKFNH~
## 3 angiotensin-converting enzyme 2 isoform X3 MSSSSWLLLSLVAVTAAQSTIEEQAKTFLDKFNH~
\end{verbatim}

\emph{2. Show the amino acid sequence between the 27th and 63rd amino
acid sequence for each isoform using str\_sub.}

\begin{Shaded}
\begin{Highlighting}[]
\CommentTok{\#Use str\_sub to extract positions 27 to 63 from Protein.Sequence}
\NormalTok{ace2\_isoforms }\OtherTok{\textless{}{-}}\NormalTok{ ace2\_isoforms }\SpecialCharTok{\%\textgreater{}\%}
  \FunctionTok{mutate}\NormalTok{(}\AttributeTok{seq\_27\_63 =} \FunctionTok{str\_sub}\NormalTok{(Protein.Sequence, }\DecValTok{27}\NormalTok{, }\DecValTok{63}\NormalTok{))}

\CommentTok{\# Display the extracted subsequence}
\NormalTok{ace2\_isoforms }\SpecialCharTok{\%\textgreater{}\%} \FunctionTok{select}\NormalTok{(Gene, seq\_27\_63)}
\end{Highlighting}
\end{Shaded}

\begin{verbatim}
## # A tibble: 3 x 2
##   Gene                                       seq_27_63                          
##   <chr>                                      <chr>                              
## 1 angiotensin-converting enzyme 2 isoform X1 TFLDKFNHEAEDLFYQSSLASWNYNTNITEENVQ~
## 2 angiotensin-converting enzyme 2 isoform X2 TFLDKFNHEAEDLFYQSSLASWNYNTNITEENVQ~
## 3 angiotensin-converting enzyme 2 isoform X3 TFLDKFNHEAEDLFYQSSLASWNYNTNITEENVQ~
\end{verbatim}

\emph{3. Print a statement using cat with paste or sprintf of the first
thirty amino acids for each isoform with a new line after each isoform
and sequence listing. cat is necessary to use in combination with paste
or sprintf to output concatenation with a new line.}

\begin{Shaded}
\begin{Highlighting}[]
\FunctionTok{cat}\NormalTok{(}\StringTok{"First 30 amino acids of each isoform:}\SpecialCharTok{\textbackslash{}n\textbackslash{}n}\StringTok{"}\NormalTok{)}
\end{Highlighting}
\end{Shaded}

\begin{verbatim}
## First 30 amino acids of each isoform:
\end{verbatim}

\begin{Shaded}
\begin{Highlighting}[]
\CommentTok{\#for loop searches for string mathches}
\ControlFlowTok{for}\NormalTok{ (i }\ControlFlowTok{in} \DecValTok{1}\SpecialCharTok{:}\FunctionTok{nrow}\NormalTok{(ace2\_isoforms)) \{}
\NormalTok{  gene\_name }\OtherTok{\textless{}{-}}\NormalTok{ ace2\_isoforms}\SpecialCharTok{$}\NormalTok{Gene[i]}
\NormalTok{  seq30 }\OtherTok{\textless{}{-}} \FunctionTok{str\_sub}\NormalTok{(ace2\_isoforms}\SpecialCharTok{$}\NormalTok{Protein.Sequence[i], }\DecValTok{1}\NormalTok{, }\DecValTok{30}\NormalTok{)}
  \FunctionTok{cat}\NormalTok{(}\FunctionTok{sprintf}\NormalTok{(}\StringTok{"\%s: \%s}\SpecialCharTok{\textbackslash{}n\textbackslash{}n}\StringTok{"}\NormalTok{, gene\_name, seq30))}
\NormalTok{\}}
\end{Highlighting}
\end{Shaded}

\begin{verbatim}
## angiotensin-converting enzyme 2 isoform X1: MSSSSWLLLSLVAVTAAQSTIEEQAKTFLD
## 
## angiotensin-converting enzyme 2 isoform X2: MSSSSWLLLSLVAVTAAQSTIEEQAKTFLD
## 
## angiotensin-converting enzyme 2 isoform X3: MSSSSWLLLSLVAVTAAQSTIEEQAKTFLD
\end{verbatim}

\end{document}
